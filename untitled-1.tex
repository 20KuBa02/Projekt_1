\documentclass[a4paper, 12p] {article}
\usepackage{graphicx}      % umożliwia wstawianie obrazów i grafik
\usepackage{amsmath}       % rozszerza możliwości edycji wzorów matematycznych
\usepackage{amsfonts}      % zapewnia dostęp do matematycznych symboli i alfabetów
\usepackage{amssymb}       % zawiera dodatkowe matematyczne symbole
\usepackage{geometry}      % umożliwia ustawianie marginesów i rozmiaru strony
\usepackage{float}         % umożliwia precyzyjne ustawianie pozycji obiektów np. figur i tabel
\usepackage{hyperref}      % tworzy odnośniki w dokumencie
\usepackage{color}         % umożliwia definiowanie i używanie kolorów
\usepackage{listings}      % umożliwia wstawianie kodu źródłowego z różnych języków programowania
\usepackage{booktabs}      % umożliwia tworzenie tabel z różnymi kolorami i grubością linii pomiędzy wierszami
\usepackage{caption}       % umożliwia kontrolowanie sposobu formatowania podpisów pod rysunkami i tabelami
\usepackage{subcaption}    % umożliwia tworzenie podobrazów i sub-tabel
\usepackage{siunitx}       % umożliwia formatowanie jednostek miar i wartości numerycznych
\usepackage{babel}         % dostarcza automatycznego tłumaczenia słów kluczowych w zależności od języka dokumentu
\usepackage{natbib}        % umożliwia bibliografie w stylu 'autor-data'
\usepackage{fancyhdr}      % umożliwia dodawanie nagłówków i stopk z dowolnym tekstem
\usepackage{datetime}      % umożliwia formatowanie daty i czasu
\usepackage{tikz}          % umożliwia rysowanie grafik wektorowych i schematów w LaTeXie
\usepackage{pgfplots}      % umożliwia tworzenie wykresów funkcji matematycznych w LaTeXie
\usepackage{tocloft}
\usepackage{polski}
\usepackage{graphicx}      % Required for inserting images
\usepackage{multicol}
\usepackage[utf8]{inputenc}
\usepackage{svg}
\usepackage{listings}
\usepackage{hyperref}
\usepackage{attachfile}
\usepackage{xcolor}
\usepackage{colortbl}
\usepackage{multirow}
%%%%%%%%%%%%%%%%%%%%%%%%%%%%%%%%%%%%%%%%%%%%%%%%

% start
\begin{document}

% strona tytułowa
\begin{center} 
\rule{\textwidth}{1.0pt} \\
\vspace{0.5cm}
    \includegraphics[width=.6\linewidth]{GIK/logo_WGiK.png}
\vspace{0.2cm} \\
	\huge \textsc{Transformacje}
\vspace{1.0cm} \\  
	\Large \textsc{Informatyka Geodezyjna II, Rok 2022/2023 \\ grupa 2, Projekt 1}
\vspace{0.4cm}\\
	\textsc{Jakub Tokarski 319392 ; Aleksandra Ślwińska 319389} 
\vspace{0.3cm}\\
	\textsc{Wydział Geodezji i Kartografii,Politechnika Warszawska} \\
        \textsc{Warszawa, 1 Maja 2023}
\end{center} 
\rule{\textwidth}{1.0pt}

\newpage   
\section{Wstęp}
\subsection{Cel ćwiczenia}
Napisanie skryptu, który implementuje geodezyjne funkcje transformacyjne. 

\subsection{Polecenie}
Napisz program w języku Python, który będzie posiadał klasę zawierającą metody implementujące podane niżej transformacje współrzędnych geodezyjnych. \\
1) XYZ na BLH \\
2) BLH na XYZ \\
3) XYZ na NEUp \\ 
4) BL (GRS80, WGS84, ew. Krasowski) na 2000 \\
5) BL (GRS80, WGS84, ew. Krasowski) na 1992 \\

Program powinien umożliwiać podawanie argumentów przy wywołaniu za pomocą biblioteki argparse. Powinien obsługiwać transformację wielu współrzędnych zapisanych w pliku tekstowym przekazywanym jako argument i tworzyć plik wynikowy.

\subsection{Narzędzia}
Aby korystać z napisanego skryptu należy mieć w swoim komputerze następujące programy/systemy: \\
-Windows 10 (oraz nowsze) \\ 
-Python 3.8 (oraz nowsze) \\ 
-Biblioteki pobrane w programie: argparse, numpy, math, os \\ 
-Programy: Program stworzony na 3 semestrze podczas zajęć z Geodezji Wyższej \\


\newpage
\section{Program}
\input{program.tex}
\subsection{Opis działania programu}
Program który napisaliśmy służy do przeliczania następujacych współrzędnych:\\
- XYZ (geocentryczne) -> BLH (elipsoidalne fi, lambda, h)\\
-BLH -> XYZ\\
-XYZ -> NEU\\
-BL(GRS80, WGS84, ew. Krasowski) -> Gaussa-Krugera\\
-BL(GRS80, WGS84, ew. Krasowski) -> 2000\\
-BL(GRS80, WGS84, ew. Krasowski) -> 1992\\
Program przyjmuje plik txt, a następnie przelicza współrzędne do zadanego układu.

\newpage
\section{Wnioski}
\large Stworzone przez nas \textbf{\href{https://github.com/20KuBa02/Projekt_1}
{REPOZYTORIUM}}\\


\bibliography{bibliografia}
@Misc{str1,
    author       = {{ASG EUPOS}},
    howpublished = {\url{www.asgeupos.pl/index.php?wpg_type=tech_transf&sub=xyz_blh}},
    note         = {Accesed: 2023-04-21},
    title        = {{Strona systemu ASG-EUPOS}},
    year         = {2023},
}

@Misc{NCEI2020,
    author       = {{NCEI}},
    howpublished = {\url{www.ngdc.noaa.gov/geomag/WMM/DoDWMM.shtml}},
    note         = {Dostęp: 2020-09-30},
    title        = {{NCEI Geomagnetic Modeling Team and British Geological Survey. 2019. World Magnetic Model 2020}},
    year         = {2020},
    doi          = {10.25921/11v3-da71},
}

@Misc{str3,
    author       = {{GEONET}},
    howpublished = {\url{www.geonet.net.pl/images/2002_12_uklady_wspolrz.pdf}},
    note         = {Accesed: 2023-04-21},
    title        = {{Strona portalu GEONET}},
    year         = {2023},
}

@Book{Borkowski.Przybylski2015,
    title    = {Książka kucharska LaTeX},
    year     = {2015},
    author   = {Marcin Borkowski and Bartłomiej Przybylski},
    publisher = {Springer Wien New York},
}




\end{document}